\paragraph{Numeric r=3 (True)}
\begin{quote}\small
\textbf{ELICIT (ASK/EXPLORE)} Can you explain how you arrived at the area being 28?
\\[0.2em]
\textbf{REVIEW+QUIZ (SUMMARIZE)} Can you explain why plugging \( r = 3 \) into the formula \( \pi r^2 \) is a convincing criterion for determining the area of a circle?
\\[0.2em]
\end{quote}
\paragraph{RAG mini-lesson}
\begin{quote}\small
\textbf{ELICIT (ASK/EXPLORE)} Before I look things up: what do you already know about this topic, and what do you most want to understand (e.g., ‘why it reduces hallucinations’, ‘how to wire it in code’, ‘tradeoffs vs. fine-tuning’)?
\\[0.2em]
\textbf{RETRIEVE (VERIFY)} Here are relevant snippets. Read them, then draft your explanation. Top passages for: Explain Retrieval-Augmented Generation and why it reduces hallucinations. - score=8.699 | question. One way to mitigate this is to do a traditional text s…
\\[0.2em]
\textbf{REVIEW+QUIZ (SUMMARIZE)} Your explanation is clear and captures the core idea of RAG. To improve, you could explicitly mention how RAG reduces hallucinations by integrating retrieved information, which directly supports the model's responses. Quick check (answer in…
\\[0.2em]
\end{quote}